% Документация проекта "SonoTracker" - веб-приложение для мониторинга сна
% Оформление по ГОСТ Р 7.32-2017
\documentclass[14pt, a4paper]{extarticle}

% ================ ПАКЕТЫ ================
\usepackage[T2A]{fontenc}
\usepackage[utf8]{inputenc}
\usepackage[english, russian]{babel}

% Геометрия страницы по ГОСТ
\usepackage[left=30mm, right=15mm, top=20mm, bottom=20mm]{geometry}

% Настройки переноса слов
\sloppy
\hyphenpenalty=1000
\tolerance=2000

% Шрифт Times-подобный
\usepackage{mathptmx}

% Межстрочный интервал 1.5
\usepackage{setspace}
\onehalfspacing

% Абзацный отступ 1.25 см
\usepackage{indentfirst}
\setlength{\parindent}{1.25cm}

% Оформление заголовков разделов
\usepackage{titlesec}
\titleformat{\section}
  {\normalfont\fontsize{14}{16}\bfseries\centering}
  {\thesection}{1em}{}
\titleformat{\subsection}
  {\normalfont\fontsize{14}{16}\bfseries}
  {\thesubsection}{1em}{}
\titleformat{\subsubsection}
  {\normalfont\fontsize{14}{16}\bfseries}
  {\thesubsubsection}{1em}{}

% Нумерация страниц внизу по центру
\usepackage{fancyhdr}
\pagestyle{fancy}
\fancyhf{}
\fancyfoot[C]{\thepage}
\renewcommand{\headrulewidth}{0pt}

% Для работы с изображениями
\usepackage{graphicx}
\usepackage{float}
\graphicspath{{./}{Скрины/}{Скриншоты/}}

% Для листингов кода
\usepackage{listings}
\usepackage{xcolor}

\definecolor{codegreen}{rgb}{0,0.6,0}
\definecolor{codegray}{rgb}{0.5,0.5,0.5}
\definecolor{codepurple}{rgb}{0.58,0,0.82}
\definecolor{backcolour}{rgb}{0.95,0.95,0.92}

\lstdefinestyle{mystyle}{
    backgroundcolor=\color{backcolour},   
    commentstyle=\color{codegreen},
    keywordstyle=\color{blue},
    numberstyle=\tiny\color{codegray},
    stringstyle=\color{codepurple},
    basicstyle=\ttfamily\footnotesize,
    breakatwhitespace=false,         
    breaklines=true,                 
    captionpos=t,                    
    keepspaces=true,                 
    numbers=left,                    
    numbersep=5pt,                  
    showspaces=false,                
    showstringspaces=false,
    showtabs=false,                  
    tabsize=2,
    frame=single
}
\lstset{style=mystyle}

% Для гиперссылок
\usepackage{hyperref}
\hypersetup{
    colorlinks=true,
    linkcolor=black,
    urlcolor=blue,
    citecolor=black
}

% Для подписей к рисункам и таблицам
\usepackage{caption}
\captionsetup[figure]{name=Рисунок, labelsep=endash, justification=centering}
\captionsetup[table]{name=Таблица, labelsep=endash, justification=centering}

% Для перечислений
\usepackage{enumitem}
\setlist{nolistsep}

% ================ НАЧАЛО ДОКУМЕНТА ================
\begin{document}

% ================ ТИТУЛЬНЫЙ ЛИСТ ================
\begin{titlepage}
\begin{center}

\textbf{Государственное бюджетное общеобразовательное учреждение\\
города Москвы «Школа № 2200»}

\vspace{2cm}

{\fontsize{18}{22}\selectfont\bfseries
Разработка веб-приложения\\[0.3cm]
для мониторинга и анализа сна}

\vspace{0.8cm}

{\fontsize{32}{38}\selectfont\bfseries\itshape
SONOTRACKER}

\vspace{0.5cm}

{\fontsize{16}{20}\selectfont
цифровой дневник сна\\[0.2cm]
с визуализацией данных}

\vspace{0.5cm}
\rule{0.6\textwidth}{1pt}

\vspace{2cm}

\hfill
\begin{minipage}{0.5\textwidth}
\textbf{Выполнили:}\\
ученица 10 «С» класса\\
Савкина Арина Дмитриевна\\[0.5cm]
ученица 10 «С» класса\\
Воронова Дарья Михайловна\\[1cm]
\textbf{Руководитель:}\\
педагог ГБОУ Школа № 2200\\
Неумоин Павел Дмитриевич
\end{minipage}

\vfill

Москва, 2026

\end{center}
\end{titlepage}

% ================ СОДЕРЖАНИЕ ================
\tableofcontents
\newpage

% ================ ВВЕДЕНИЕ ================
\section*{Введение}
\addcontentsline{toc}{section}{Введение}

\textbf{Актуальность:} качество сна оказывает прямое влияние на здоровье, работоспособность и эмоциональное состояние человека. В современном мире нарушения сна становятся всё более распространённой проблемой, особенно среди молодёжи. По данным Всемирной организации здравоохранения, до 45\% населения испытывают те или иные проблемы со сном.

\textbf{Проблема:} многие люди не отслеживают свой режим сна и не осознают влияние его нарушений на качество жизни. Существующие решения (смарт-часы, браслеты) требуют финансовых затрат и не всегда доступны.

\textbf{Объект исследования:} процесс мониторинга и анализа сна человека.

\textbf{Предмет исследования:} веб-технологии для создания трекера сна с ручным вводом данных.

\textbf{Гипотеза:} использование трекеров сна повышает осознанность пользователя в отношении режима сна и способствует формированию здоровых привычек.

\textbf{Цель:} разработка веб-приложения для мониторинга сна с возможностью ввода данных, расчёта показателей, визуализации результатов и интеллектуальной поддержкой на основе нейросети.

\textbf{Задачи:}
\begin{enumerate}
    \item Изучить физиологические основы сна, его стадии и влияние на здоровье.
    \item Проанализировать принципы работы существующих трекеров сна.
    \item Разработать веб-интерфейс с формой ввода данных о сне.
    \item Реализовать алгоритм расчёта продолжительности и качества сна.
    \item Создать систему визуализации данных в виде диаграмм.
    \item Интегрировать нейросеть YandexGPT для создания ИИ-помощника.
    \item Провести тестирование работоспособности приложения.
\end{enumerate}

% ================ ТЕОРЕТИЧЕСКАЯ ЧАСТЬ ================
\section{Теоретическая часть}

\subsection{Физиология сна}

Сон --- это естественное физиологическое состояние, характеризующееся пониженной реакцией на окружающий мир. Сон необходим для восстановления физических и психических сил организма.

\textbf{Фазы сна:}
\begin{itemize}
    \item \textbf{Медленный сон (Non-REM)} --- включает 4 стадии, от дремоты до глубокого сна. Занимает около 75--80\% общего времени сна.
    \item \textbf{Быстрый сон (REM)} --- фаза с быстрым движением глаз, во время которой происходят сновидения. Занимает около 20--25\% времени сна.
\end{itemize}

Полный цикл сна длится примерно 90--110 минут и повторяется 4--6 раз за ночь.

\subsection{Проблемы сна в современном обществе}

Последствия нарушений сна включают:
\begin{itemize}
    \item снижение концентрации внимания;
    \item ухудшение памяти;
    \item повышенную утомляемость;
    \item эмоциональную нестабильность;
    \item ослабление иммунитета;
    \item повышенный риск хронических заболеваний.
\end{itemize}

\subsection{Трекеры сна: устройство и принципы работы}

Трекер сна --- это электронное устройство или программное приложение, предназначенное для сбора и анализа данных о сне пользователя.

\textbf{Виды трекеров сна:}
\begin{itemize}
    \item \textbf{Носимые устройства} --- смарт-часы, фитнес-браслеты с датчиками движения и пульса;
    \item \textbf{Мобильные приложения} --- используют акселерометр и микрофон смартфона;
    \item \textbf{Бесконтактные системы} --- датчики под матрасом или рядом с кроватью.
\end{itemize}

\textbf{Параметры, фиксируемые трекерами:}
\begin{itemize}
    \item длительность сна;
    \item фазы сна;
    \item количество пробуждений;
    \item частота сердечных сокращений;
    \item качество сна (оценка).
\end{itemize}

Использование неинвазивных датчиков делает трекеры доступными и безопасными для повседневного применения, в отличие от клинической полисомнографии.

% ================ ПРАКТИЧЕСКАЯ ЧАСТЬ ================
\section{Практическая часть}

\subsection{Выбор технологий}

Для разработки веб-приложения были выбраны следующие технологии:

\begin{table}[H]
\centering
\caption{Используемые технологии}
\begin{tabular}{|l|l|l|}
\hline
\textbf{Технология} & \textbf{Назначение} & \textbf{Версия} \\
\hline
HTML5 & Структура страниц & 5 \\
\hline
CSS3 & Визуальное оформление & 3 \\
\hline
JavaScript & Логика обработки данных & ES6+ \\
\hline
Node.js & Серверная часть & 18+ \\
\hline
YandexGPT & Нейросеть для ИИ-помощника & API v1 \\
\hline
\end{tabular}
\end{table}

\subsection{Структура веб-проекта}

Проект состоит из следующих файлов:

\begin{itemize}
    \item \texttt{index.html} --- главная страница с формой ввода данных и блоком отображения результатов;
    \item \texttt{sleep-phases.html} --- информация о фазах сна;
    \item \texttt{sleep-science.html} --- научные данные о физиологии сна;
    \item \texttt{sleep-statistics.html} --- статистика и визуализация данных пользователя;
    \item \texttt{sleep-tips.html} --- рекомендации по улучшению качества сна;
    \item \texttt{ai-helper.html} --- страница ИИ-помощника <<Соня>>;
    \item \texttt{server/server.js} --- серверная часть для интеграции с YandexGPT.
\end{itemize}

\subsection{Алгоритм обработки данных}

Алгоритм работы приложения:
\begin{enumerate}
    \item JavaScript считывает введённые пользователем значения (время засыпания, время пробуждения).
    \item Рассчитывается общая продолжительность сна.
    \item На основе введённых параметров формируется оценка качества сна.
    \item Результаты выводятся на экран в числовом и графическом виде.
\end{enumerate}

\subsection{Функциональные возможности сайта}

Разработанный сайт позволяет:
\begin{itemize}
    \item вводить данные о времени засыпания и пробуждения;
    \item анализировать режим сна пользователя;
    \item визуализировать результаты в виде диаграмм;
    \item отслеживать изменения показателей при соблюдении режима;
    \item использовать сайт как цифровой дневник сна;
    \item получать рекомендации по улучшению качества сна.
\end{itemize}

\subsection{Обоснование выбора подхода}

Ручной ввод данных был выбран как начальный этап разработки, так как он:
\begin{itemize}
    \item упрощает тестирование алгоритмов анализа сна;
    \item не требует подключения аппаратной части;
    \item позволяет сосредоточиться на программной логике, ИИ-функциях и интерфейсе;
    \item широко используется на ранних этапах проектирования цифровых систем.
\end{itemize}

\subsection{Интеграция нейросети YandexGPT}

Одним из ключевых нововведений проекта является интеграция нейросети YandexGPT --- современной языковой модели, разработанной компанией Яндекс. Данная интеграция позволила создать интерактивного ИИ-помощника <<Соня>> (персонаж --- сонное облачко), который значительно расширяет функциональность приложения.

\textbf{Архитектура интеграции:}
\begin{itemize}
    \item \textbf{Клиентская часть} --- JavaScript-модуль \texttt{ai-helper.js}, обеспечивающий взаимодействие с пользователем через веб-интерфейс;
    \item \textbf{Серверная часть} --- Node.js сервер, выступающий прокси между клиентом и API YandexGPT для безопасной передачи запросов;
    \item \textbf{API YandexGPT} --- облачный сервис Яндекса для генерации ответов на основе нейросети.
\end{itemize}

\textbf{Функции ИИ-помощника <<Соня>>:}
\begin{enumerate}
    \item \textbf{Интерактивный чат} --- пользователь может задать любой вопрос о сне, и облачко <<Соня>> ответит, используя контекстные знания нейросети;
    \item \textbf{Генерация сказок на ночь} --- персонализированные сказки с учётом имени ребёнка и выбранной темы (волшебный сон, путешествие по облакам, звёздная ночь и др.);
    \item \textbf{Толкование снов} --- объяснение значения снов в доступной и дружелюбной форме для детей.
\end{enumerate}

\textbf{Технические особенности:}
\begin{itemize}
    \item асинхронная обработка запросов с индикатором загрузки;
    \item система запасных ответов (fallback) при недоступности сервера;
    \item автоматическая проверка статуса подключения к ИИ;
    \item сохранение имени пользователя в localStorage для персонализации.
\end{itemize}

Использование YandexGPT делает приложение более интерактивным и привлекательным для целевой аудитории --- детей и подростков, способствуя формированию здоровых привычек сна в игровой форме.

% ================ ТЕСТИРОВАНИЕ ================
\section{Тестирование}

\subsection{Методика тестирования}

Тестирование сайта проводилось путём ручного ввода данных сна за 7 дней.

Корректность работы проверялась по следующим критериям:
\begin{itemize}
    \item правильность расчёта продолжительности сна;
    \item соответствие результатов логике алгоритма;
    \item стабильность работы интерфейса;
    \item корректность отображения диаграмм.
\end{itemize}

\subsection{Результаты тестирования}

В результате тестирования установлено:
\begin{itemize}
    \item расчёт продолжительности сна выполняется корректно во всех тестовых случаях;
    \item визуализация данных в виде диаграмм отображается без ошибок;
    \item логика JavaScript-алгоритма соответствует заданной методике обработки данных;
    \item интерфейс сайта стабильно работает при многократном вводе информации.
\end{itemize}

Разработанный веб-интерфейс может использоваться как цифровой дневник сна и инструмент первичного анализа режима сна пользователя.

% ================ РЕЗУЛЬТАТЫ ================
\section{Результаты проекта}

В ходе выполнения проекта был создан и протестирован веб-сайт <<SonoTracker>> с возможностью ручного ввода данных о сне.

\textbf{Реализованный функционал:}
\begin{itemize}
    \item форма ввода времени засыпания и пробуждения;
    \item автоматический расчёт продолжительности сна;
    \item оценка качества сна;
    \item визуализация данных в виде диаграмм;
    \item информационные страницы о фазах сна и научных исследованиях;
    \item раздел с рекомендациями по улучшению сна;
    \item ИИ-помощник <<Соня>> на базе YandexGPT;
    \item генератор персонализированных сказок на ночь;
    \item интерактивное толкование снов.
\end{itemize}

\textbf{Технические характеристики:}
\begin{itemize}
    \item 6 HTML-страниц (включая страницу ИИ-помощника);
    \item адаптивный дизайн;
    \item клиент-серверная архитектура с Node.js backend;
    \item интеграция с API YandexGPT для генерации контента;
    \item совместимость с современными браузерами.
\end{itemize}

\section{Перспективы развития}

\textbf{Где применимо:}
\begin{itemize}
    \item в образовательной среде;
    \item для самоконтроля режима сна;
    \item в качестве вспомогательного инструмента при консультациях со специалистами.
\end{itemize}

\textbf{Направления развития:}
\begin{enumerate}
    \item Интеграция с носимыми устройствами для автоматического сбора данных.
    \item Добавление серверной части для хранения истории данных.
    \item Разработка мобильного приложения.
    \item Расширение алгоритма анализа с использованием машинного обучения.
    \item Добавление функции умного будильника.
\end{enumerate}

% ================ ЗАКЛЮЧЕНИЕ ================
\section*{Заключение}
\addcontentsline{toc}{section}{Заключение}

В ходе проектной работы были изучены физиологические основы сна, его стадии и основные проблемы сна в современном обществе. Проанализированы принципы работы существующих трекеров сна, их преимущества и ограничения при бытовом использовании.

Разработан веб-интерфейс трекера сна с ручным вводом данных, реализованный с использованием HTML, CSS и JavaScript. Приложение позволяет рассчитывать продолжительность сна и визуализировать данные пользователя.

Проведено тестирование разработанного сайта, подтвердившее корректность расчётов и стабильность работы интерфейса.

\textbf{Выводы:}
\begin{enumerate}
    \item Изучены физиологические основы сна, его стадии и влияние на здоровье человека.
    \item Проанализированы принципы работы существующих трекеров сна.
    \item Разработан веб-интерфейс трекера сна с формой ввода данных и визуализацией.
    \item Успешно интегрирована нейросеть YandexGPT для создания интерактивного ИИ-помощника.
    \item Проведено тестирование, подтвердившее корректность работы приложения и ИИ-функций.
    \item Полученные результаты подтверждают гипотезу о том, что использование трекеров сна повышает осознанность пользователя в отношении режима сна.
\end{enumerate}

% ================ СПИСОК ЛИТЕРАТУРЫ ================
\section*{Список литературы}
\addcontentsline{toc}{section}{Список литературы}

\begin{enumerate}
    \item Вейн А. М. Сон и бодрствование человека. --- М.: Медицина, 2019. --- 320 с.
    
    \item Смирнов С. Д. Психология сна: учебное пособие. --- СПб.: Питер, 2020. --- 256 с.
    
    \item Иванов И. И. Фазы сна и их значение // Вестник медицины. --- 2021. --- № 4. --- С. 45--50.
    
    \item MDN Web Docs. HTML, CSS, JavaScript guides [Электронный ресурс]. --- Режим доступа: \url{https://developer.mozilla.org/ru/} (дата обращения: 20.01.2026).
    
    \item World Health Organization. Sleep and health [Электронный ресурс]. --- Режим доступа: \url{https://www.who.int/} (дата обращения: 15.01.2026).
\end{enumerate}

% ================ ПРИЛОЖЕНИЕ ================
\section*{Приложение}
\addcontentsline{toc}{section}{Приложение}

\subsection*{Приложение А --- Скриншоты веб-приложения}

\begin{figure}[H]
    \centering
    \includegraphics[width=0.9\textwidth]{2026-01-29_20-05-54.png}
    \caption{ИИ-помощник <<Соня>> --- главный экран с функциями чата, сказок и толкования снов}
    \label{fig:ai-helper}
\end{figure}

\begin{figure}[H]
    \centering
    \includegraphics[width=0.9\textwidth]{2026-01-29_20-06-07.png}
    \caption{Страница <<Фазы сна>> --- интерактивное объяснение циклов сна}
    \label{fig:phases}
\end{figure}

\begin{figure}[H]
    \centering
    \includegraphics[width=0.9\textwidth]{2026-01-29_20-06-15.png}
    \caption{Секреты крепкого сна --- рекомендации по улучшению качества сна}
    \label{fig:tips}
\end{figure}

\begin{figure}[H]
    \centering
    \includegraphics[width=0.9\textwidth]{2026-01-29_20-06-22.png}
    \caption{Статистика сна --- визуализация достижений пользователя}
    \label{fig:stats}
\end{figure}

\begin{figure}[H]
    \centering
    \includegraphics[width=0.9\textwidth]{2026-01-29_20-06-41.png}
    \caption{Генератор сказок на ночь --- персонализированные истории от YandexGPT}
    \label{fig:story-generator}
\end{figure}

\end{document}
